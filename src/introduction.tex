\part{Introduction}

\textbf{C语言程序设计}

(1) 模块化程序设计、模块与函数、模块的设计原则

(2) 面向任务的程序设计

(3) 面向对象的程序设计

(4) 程序设计语言

(5) C语言的特点

\textbf{算法}

(1) 算法的性质

(2) 与算法密切相关的——语句

(3) 算法的描述

(4) 常用算法

\textbf{程序设计中的数据以及数据类型、数据的存放}

(1) 字节

(2) 字

(3) 常量

(4) 变量、变量的作用域以及分类

\textbf{C语言程序的特点}

\textbf{C语言字符集}

\textbf{C语言关键字}

\textbf{C语言运算符及其优先级}

\textbf{C语言表达式}

\textbf{C语言语句}

(1) 简单语句、空语句

(2) 复合语句

(3) if语句

(4) switch语句

(5) while语句

(6) do-while语句

(7) for语句

(8) break、continue和goto语句

\textbf{数据的输入输出}

\textbf{C语言函数}

\textbf{C语言程序的编译预处理}

\textbf{C语言库函数}


\chapter{应该从一开始就培养良好的编程风格和习惯}


现在还不太会或者不习惯写独立函数。仔细想想,回顾一下以前看过的C语言教程,很多示例、功能代码都写在main函数中,输出的系列信息字符串也是直接写在代码中,虽然这样比较简单,对于初学者来说,也比较容易理解,但这对从一开始就培养良好编程习惯是很不好的。

在多数大学,所学的第一个语言就是C语言。学第一个语言养成的风格和习惯,对未来的学习和工作都会产生重要影响,因此,从一开始就应该培养良好的编程风格和习惯。写教程的大牛们也应该注意这一点。

良好的可测试性和可维护性是代码的基本要求,应该一开始就培养这方面的意识。

\textbf{程序设计}

计算机依据程序运行,为了让计算机按人的意图处理事务,必须要预先设计好完成各种任务的程序,并且预先将它们存放在存储器中。

\textbf{算法}

函数之所以对于程序设计来说很重要,其中的一个原因是函数提供了算法实现的基础。算法本身是抽象的策略,通常用自然语言表达,而函数是以某种程序设计语言表示的算法的具体实现。当要将算法作为程序的一部分实现时,通常要写一个函数来执行算法,而该函数也可以调用其它函数处理它的一部分工作。

在设计程序来解决问题时,随着问题越来越复杂,需要周密地考虑才能找到解决的策略,所以在写最终的程序前,需要考虑多个策略而不是一个策略。通过用多种解决方案解决一个问题,可以明白如何比较不同的策略,以及如何选取最佳的策略。

\begin{figure}[!ht]
\centering
\includegraphics[scale=0.5]{cs.png}
\label{cs}
\end{figure}



































